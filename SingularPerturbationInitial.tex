\documentclass{article}
\usepackage{fontspec}
\usepackage{xeCJK}
\usepackage{graphicx}
\usepackage{hyperref}
\usepackage{amsmath}
\usepackage{amssymb}
\usepackage{ulem}
\usepackage{geometry}
\usepackage{color}
\usepackage{fancyhdr}
\usepackage{lastpage}
\usepackage{amsthm}

% 英文正文字体
\setmainfont{Times New Roman}
\setsansfont{Arial}
\setmonofont{Consolas}

% 中文字体
\setCJKmainfont{SimSun}
\setCJKsansfont{Microsoft YaHei}
\setCJKmonofont{FangSong}

\title{奇异摄动理论入门学习报告}
\author{赵智阳}
\date{\today}

\begin{document}
\maketitle

\section{核心思想理解}
奇异摄动理论是研究含小参数系统中、因参数趋于零导致解的性质发生突变(如边界层、内部层等急剧变化)的多尺度分析方法。
\subsection{正则摄动与奇异摄动}
正则摄动与奇异摄动是处理含小参数问题的两种不同方法,核心区别在于解的渐近行为是否一致收敛于退化问题的解。

正则摄动的核心特征是小参数$\epsilon$的影响是温和、全局一致的。即当$\epsilon \to 0$时,解$y(x;\epsilon)$在整个区域上一致收敛于退化问题$(\epsilon = 0)$的解$y_0(x)$。求解时可直接将解展开为$\epsilon$的幂级数$y=y_0+\epsilon y_1+\epsilon^2y_2+\dots$,再代回问题逐项求解。常见问题如有弹簧刚度微调后的振动频率计算。

而奇异摄动的核心特征:小参数$\epsilon$导致问题的定性性质发生剧烈、局部突变。当$\epsilon \to 0$时,解$y(x;\epsilon)$非一致收敛。在退化问题$(\epsilon = 0)$中解的某些特征(如边界条件、方程阶数/类型)会丢失,从而在局部区域产生薄层,层内解发生剧烈变化。求解时需要采用匹配渐近展开、多重尺度等特殊技巧,针对不同尺度区域分别展开并匹配。物理实例有小粘性流体在高雷诺数下绕物体流动,即普朗特边界层理论。
\subsection{边界层理论}
面对无粘理论失效、全粘性理论难以求解的现状,德国力学家路德维希·普朗特于1904年开创边界层理论,系统解决了高雷诺数下流体绕物体流动时,无粘理论与实验观测间的根本矛盾。
\begin{enumerate}
    \item 无粘理论失效:欧拉方程($v=0$)允许流体在物面“滑移”,无法解释壁面无滑移条件及由此产生的摩擦阻力。
    \item 全粘性理论难以求解:纳维-斯托克斯方程(N-S方程)在全域求解极端困难(高雷诺数时数值刚性极大)。
\end{enumerate}

普朗特发现,小粘性($\nu \to 0$ 或 $Re \to \infty$)的影响并非全局,而是高度局域化在物体表面附近一个极薄的区域内,该区域被称为\textbf{边界层}。基于此,他提出了两个关键思想:

\begin{enumerate}
    \item \textbf{尺度分离}:外层区域下,惯性力主导,粘性力可忽略,可用简化后的欧拉方程描述。内层区域下,粘性力与惯性力\textbf{同等重要},必须同时保留。
    \item \textbf{尺度变换(坐标拉伸)}引入边界层坐标变换以解析内层行为。对于平板流动,法向坐标$y$被拉伸为$Y = \frac{y}{\delta(Re)}, \quad \text{其中边界层厚度} \ \delta \sim Re^{-1/2} \ \text{(对于层流)}$。通过系统的量级分析,纳维-斯托克斯方程在边界层内可简化为更易求解的\textbf{普朗特边界层方程}。
\end{enumerate}

\section{案例分析}
考察二维常微分方程
\[\varepsilon y''(x) + y'(x) = 0, \quad 0 < x < 1\]
\[y(0) = 0, \quad y(1) = 1\]
其中 $0 < \varepsilon \ll 1$ 为小参数,且有精确解
\[
    y_{\text{exact}}(x) = \frac{1 - e^{-x/\varepsilon}}{1 - e^{-1/\varepsilon}}
\]
\subsection{匹配渐进展开法}
方程中一阶项系数 $a(x) = 1 > 0$,根据边界层理论,当 $a(x) > 0$ 时,边界层出现在\textbf{左端点} $x=0$ 附近。这意味着:
\begin{itemize}
    \item 外解(远离边界层的区域)应满足右边界条件 $y(1)=1$
    \item 内解(边界层内)负责满足左边界条件 $y(0)=0$
\end{itemize}
设外解展开为:
\[
y_{\text{out}}(x) = y_0(x) + \varepsilon y_1(x) + O(\varepsilon^2)
\]
代入$\varepsilon=0$,有退化方程
\[
y_0'(x) = 0
\]
解得:
\[
y_0(x) = C
\]
外解匹配右边界条件 $y(1)=1$,得
\[
    y_{\text{out}}(x) = 1 + O(\varepsilon)
\]

在左边界层内,引入拉伸坐标(内变量)
\[
    X = \frac{x}{\varepsilon^p}
\]
其中 $p>0$ 为待定尺度指数。令 $y(x) = Y(X)$,则有
\[
\frac{d}{dx} = \varepsilon^{-p} \frac{d}{dX}, \quad 
\frac{d^2}{dx^2} = \varepsilon^{-2p} \frac{d^2}{dX^2}
\]
代入原方程
\[
\varepsilon^{1-2p} Y''(X) + \varepsilon^{-p} Y'(X) = 0
\]
令
\[
1-2p = -p \quad \Rightarrow \quad p = 1
\]
因此边界层厚度为 $O(\varepsilon)$,内变量为
\[
X = \frac{x}{\varepsilon}
\]
此时方程简化为
\[
    Y''(X) + Y'(X) = 0
    \label{eq:inner_eq}
\]
设内解展开为
\[
Y_{\text{in}}(X) = Y_0(X) + \varepsilon Y_1(X) + O(\varepsilon^2)
\]
零阶内方程通解为
\[
Y_0(X) = A + B e^{-X}
\]
其中 $A, B$ 为待定常数。

左边界条件 $y(0)=0$ 转换为 $Y_0(0)=0$,代入得:
\[
A + B = 0 \quad \Rightarrow \quad B = -A
\]
因此
\[
    Y_0(X) = A(1 - e^{-X})
    \label{eq:inner_sol}
\]

下面应用匹配原理,即内解的外极限应等于外解的内极限。

\begin{itemize}
    \item 外解的内极限:当 $x \to 0^+$,由外层方程解得 $y_{\text{out}} \to 1$
    \item 内解的外极限:当 $X \to \infty$,由内层方程解得 $Y_0(X) \to A$
\end{itemize}

令二者相等$A = 1$,代入内层解,得完整的零阶内解
\[
Y_0(X) = 1 - e^{-X}
\]

复合展开的一般形式为:
\[
y_{\text{comp}}(x) = y_{\text{out}}(x) + Y_{\text{in}}(X) - y_{\text{match}}
\]
其中 $y_{\text{match}}$ 为内外解的共同部分(匹配值)。本例中:
\begin{align*}
    y_{\text{out}}(x) &= 1 \\
    Y_{\text{in}}(X) &= 1 - e^{-X} \\
    y_{\text{match}} &= 1
\end{align*}
因此零阶复合解为:
\[
    y_{\text{approx}}(x) = 1 + (1 - e^{-x/\varepsilon}) - 1 = 1 - e^{-x/\varepsilon}
\]

\begin{figure}[htbp]
    \centering
    \includegraphics[width=0.8\textwidth]{SingularPerturbationFigure_1.png}
    \caption{python模拟下的解图像,$\epsilon=0.05$}
\end{figure}
\begin{figure}[htbp]
    \centering
    \includegraphics[width=0.8\textwidth]{SingularPerturbationFigure_2.png}
    \caption{边界层内部与近似误差对数图}
\end{figure}

\subsection{结果分析}
比较精确解$y_{\text{exact}}(x)$和近似解$y_{\text{approx}}(x)$:
\begin{itemize}
    \item 当 $\varepsilon \to 0$ 时,$e^{-1/\varepsilon} \to 0$,精确解简化为 $1 - e^{-x/\varepsilon}$,与我们的渐近解一致
    \item 在边界层内 ($x \sim O(\varepsilon)$),$e^{-x/\varepsilon} = O(1)$,解从 0 快速上升到 1
    \item 在边界层外 ($x \gg \varepsilon$),$e^{-x/\varepsilon} \approx 0$,解近似为常数 1,与退化方程解一致
\end{itemize}

\section{附录:Python代码}
\begin{verbatim}
import numpy as np
import matplotlib.pyplot as plt
from scipy.integrate import solve_bvp

#参数
epsilon = 0.05

#精确解
x_exact = np.linspace(0, 1, 1000)
y_exact = (1 - np.exp(-x_exact/epsilon)) / (1 - np.exp(-1/epsilon))

#奇异摄动近似解
y_approx = 1 - np.exp(-x_exact/epsilon)

#退化方程解
y_reduced = np.ones_like(x_exact)

#数值求解原方程
def ode(x, y):
    return np.vstack((y[1], -y[1]/epsilon))

def bc(ya, yb):
    return np.array([ya[0], yb[0]-1])

x_num = np.linspace(0, 1, 100)
y_num = np.zeros((2, x_num.size))
y_num[0] = x_num  # 初始猜测

sol = solve_bvp(ode, bc, x_num, y_num, max_nodes=10000)
x_sol = sol.x
y_sol = sol.y[0]

# 绘图
plt.figure(figsize=(10, 6))
plt.plot(x_exact, y_exact, 'k-', linewidth=2, label='Exact Solution')
plt.plot(x_exact, y_approx, 'r--', linewidth=2, label='Singular Perturbation Approx')
plt.plot(x_exact, y_reduced, 'b:', linewidth=2, label='Reduced Solution (ε=0)')
plt.plot(x_sol, y_sol, 'go', markersize=4, label='Numerical Solution (solve_bvp)', alpha=0.6)

plt.xlabel('x', fontsize=12)
plt.ylabel('y', fontsize=12)
plt.title(f'Boundary Layer Behavior: ε = {epsilon}', fontsize=14)
plt.legend(loc='best', fontsize=10)
plt.grid(True, alpha=0.3)
plt.axvline(x=5*epsilon, color='gray', linestyle='--', alpha=0.5, label='~5ε (Boundary Layer Width)')
plt.legend()

# 子图:边界层细节
plt.figure(figsize=(10, 4))
plt.subplot(1, 2, 1)
x_detail = np.linspace(0, 5*epsilon, 200)
y_exact_detail = (1 - np.exp(-x_detail/epsilon)) / (1 - np.exp(-1/epsilon))
y_approx_detail = 1 - np.exp(-x_detail/epsilon)
plt.plot(x_detail, y_exact_detail, 'k-', label='Exact')
plt.plot(x_detail, y_approx_detail, 'r--', label='Approx')
plt.xlabel('x (zoomed)')
plt.ylabel('y')
plt.title('Inside Boundary Layer')
plt.grid(True, alpha=0.3)
plt.legend()

plt.subplot(1, 2, 2)
error = np.abs(y_exact - y_approx)
plt.semilogy(x_exact, error, 'b-')
plt.xlabel('x')
plt.ylabel('Absolute Error')
plt.title('Approximation Error')
plt.grid(True, alpha=0.3)

plt.tight_layout()
plt.show()
\end{verbatim}
\end{document}